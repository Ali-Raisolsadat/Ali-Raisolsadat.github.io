%========================
% Document class and theme
%========================
\documentclass[8pt]{beamer}
\usetheme[progressbar=frametitle]{metropolis}
\setbeamersize{text margin left=10mm, text margin right=10mm}
\usepackage{appendixnumberbeamer} % appendix slide numbering
\setbeamertemplate{theorems}[numbered]

%========================
% Core packages
%========================
\usepackage{amsmath, amsfonts, amssymb, amsthm} % math + theorems
\usepackage{booktabs}        % professional tables
\usepackage{hyperref}        % hyperlinks
\usepackage{xcolor}          % colors
\usepackage{xspace}          % spacing for custom commands

%========================
% Algorithms
%========================
\usepackage{algorithm}
\usepackage{algpseudocode}
\newtheorem{proposition}{Proposition}
\usepackage{bbm}

%========================
% Plots and TikZ
%========================
\usepackage{pgfplots}
\usepgfplotslibrary{dateplot}
\usepackage{tikz}
\usetikzlibrary{positioning}

% ==========================================
% Professional Code Listing Setup
% ==========================================
\usepackage{listings}
\definecolor{codegreen}{rgb}{0,0.6,0}
\definecolor{codegray}{rgb}{0.5,0.5,0.5}
\definecolor{codepurple}{rgb}{0.58,0,0.82}
\definecolor{backcolour}{rgb}{0.95,0.95,0.92}

\lstdefinestyle{mystyle}{
    backgroundcolor=\color{backcolour},   
    commentstyle=\color{codegreen},
    keywordstyle=\color{magenta},
    numberstyle=\tiny\color{codegray},
    stringstyle=\color{codepurple},
    basicstyle=\ttfamily\footnotesize,
    breakatwhitespace=false,         
    breaklines=true,                 
    captionpos=b,                    
    keepspaces=true,                 
    numbers=left,                    
    numbersep=5pt,                  
    showspaces=false,                
    showstringspaces=false,
    showtabs=false,                  
    tabsize=2
}
\lstset{style=mystyle}

%========================
% Custom commands
%========================
\newcommand{\themename}{\textbf{\textsc{metropolis}}\xspace}

%========================
% Custom footline
%========================
\setbeamertemplate{footline}
{%
  \leavevmode%
  \hbox{%
  \begin{beamercolorbox}[wd=.35\paperwidth,ht=2.5ex,dp=1.5ex,center]{author in head/foot}%
    \usebeamerfont{author in head/foot}\insertshortauthor
  \end{beamercolorbox}%
  \begin{beamercolorbox}[wd=.3\paperwidth,ht=2.5ex,dp=1ex,center]{title in head/foot}%
    \usebeamerfont{title in head/foot}\insertshorttitle
  \end{beamercolorbox}%
  \begin{beamercolorbox}[wd=.3\paperwidth,ht=2.5ex,dp=1ex,right]{date in head/foot}%
    \usebeamerfont{date in head/foot}\insertframenumber{} / \inserttotalframenumber
  \end{beamercolorbox}}%
  \vskip0pt%
}

%========================
% Beamer tweaks
%========================
\setbeamertemplate{navigation symbols}{} % remove default navigation symbols

\title{Chapter 3 - Monte Carlo Methods}
\subtitle{Monte Carlo Applications to Statistical Inference.\\ Point Estimators.}
\author{Prof. Alex Alvarez, Ali Raisolsadat}
\institute{School of Mathematical and Computational Sciences \\ University of Prince Edward Island}
\date{} % leave empty or add \today

%========================
% Begin document
%========================
\begin{document}

%-------------------
% Title frame
%-------------------
\maketitle

%-------------------
% Slide 1: Standard Error of an Estimator
%-------------------
\begin{frame}{Standard Error of an Estimator}
In this lecture we will see how to use Monte Carlo methods to estimate the standard error of an estimator.

\vspace{3mm}

A reminder that our framework consists on having some observations $x=(x_1,x_2,...,x_n)$ that are considered a random sample of a random variable $X=(X_1,X_2,...,X_n)\sim P_{\theta}$, where $\theta \in \Theta$
is unknown.

\vspace{3mm}

Let $\hat{\theta}=\hat{\theta}(X)$ be an estimator of $\theta$. One important quantity that we study on estimators is their standard deviation.
\begin{equation*}
s.e._{\theta}(\hat{\theta})=stdev\left(\hat{\theta}(X)\right)=\sqrt{Var\left(\hat{\theta}(X) \right)}
\end{equation*}
\end{frame}

%-------------------
% Slide 2: Standard Error of an Estimator
%-------------------
\begin{frame}{Standard Error of an Estimator}
In some cases we can estimate $s.e._{\theta}(\hat{\theta})$ easily.
\vspace{3mm}

For example, for estimators of the form $\hat{\theta}=\frac{1}{n} \sum_{j=1}^n X_j$ we can estimate their standard deviation as 
\begin{equation*}
\widehat{s.e.}_{\theta}(\hat{\theta})=\frac{\hat{s}}{\sqrt{n}}
\end{equation*}
where $\hat{s}$ is the sample standard deviation of the observations $(x_1,x_2,..,x_n)$.

\vspace{3mm}

(we used this in previous sections of this chapter to give a confidence interval of a Monte Carlo estimator)

\vspace{3mm}

However, for estimators $\hat{\theta}$ that are not given as a sample mean, estimating their standard error may be more complicated.
\end{frame}

%-------------------
% Slide 3: Estimating the Standard Error
%-------------------
\begin{frame}{Estimating the Standard Error}
For a given value of $\theta$, if we are able to generate $N$ samples: $\left\{x^{(j)}\right\}_{j=1,2,...,N}$ of the random variable $X$ we could estimate the standard error of the estimator $\hat{\theta}$ as:
\begin{equation*}
	\widehat{s.e.}_{\theta}(\hat{\theta})=\sqrt{\frac{1}{N} \sum_{j=1}^N\left(\hat{\theta}(x^{(j)}) -\overline{\hat{\theta}^{(\cdot)}}\right)^2}
\end{equation*}
where $\overline{\hat{\theta}^{(\cdot)}}=\frac{1}{N} \sum_{j=1}^N\hat{\theta}(x^{(j)})$

If necessary, we can do this for a range of values of $\theta$ to get a better idea of the standard error of the estimator as a function of $\theta$. 
\end{frame}

%-------------------
% Slide 4: Example
%-------------------
\begin{frame}{Example} 
Assume that we would like to estimate an unknown parameter $\theta$ from a sample of 20 random numbers that are distributed according to the uniform distribution on $[0,\theta]$.
Two unbiased estimators for $\theta$ are proposed and we would like to find out which of these two estimators has a smaller variance.

\vspace{3mm}

\textbf{Estimator 1}: $\hat{\theta}_1(X)=\frac{21}{20}\max(X_1,X_2,...X_{20})$
\vspace{3mm}

\textbf{Estimator 2}: $\hat{\theta}_2(X)=\frac{2}{N} \sum_{j=1}^{20} X_j$

\vspace{3mm}

Use Monte Carlo methods to estimate the standard error for both estimators, for values of $\theta$ on $[10,20]$. 
\end{frame}


%-------------------
% Slide 5: Estimating the Standard Error
%-------------------
\begin{frame}
\begin{algorithm}[H]
\caption{Monte Carlo Estimation of Estimator Variability}
\label{alg:generalized-sd-estimation}
\begin{algorithmic}[1]
  \State \textbf{Input:} Number of simulations $N$, sample size $m$, parameter grid $\{\theta_j\}_{j=1}^J$
  \For{$j = 1$ to $J$}
    \State Initialize empty vectors for estimator values
    \For{$i = 1$ to $N$}
      \State Generate $U_{i1}, \ldots, U_{im} \sim \text{Uniform}(0,\, \theta_j)$
      \State Compute estimators $\hat{\theta}_{1i} = a_1 \max(U_i) + b_1$ and $\hat{\theta}_{2i} = a_2 \, \overline{U_i} + b_2$
    \EndFor
    \State Compute standard deviations: $\texttt{sd1}_j = \text{sd}(\hat{\theta}_{1})$, \quad $\texttt{sd2}_j = \text{sd}(\hat{\theta}_{2})$
  \EndFor
  \State \textbf{Output:} Standard deviation profiles $\{\texttt{sd1}_j, \texttt{sd2}_j\}_{j=1}^J$
\end{algorithmic}
\end{algorithm}
\end{frame}

%-------------------
% Slide 6: Example plot
%-------------------
\begin{frame}
\begin{center}
\includegraphics[width=\textwidth]{sdestim.png}
\end{center}

\textbf{Note}: Estimator 1 would be preferable as it has a small standard error.
\end{frame}

%-------------------
% Slide 7: Homework
%-------------------
\begin{frame}{Homework} 
Assume that we would like to estimate the unknown parameter $\mu$ from a sample of 10 random numbers that are distributed according to the normal distribution $N(\mu,1)$. We don't have the full information about these observations, we only have the min, median and max.

Two unbiased estimators for $\mu$ are proposed and we would like to find out which of these two estimators has a smaller variance.

\vspace{3mm}

Estimator 1: $\displaystyle{\hat{\mu}_1(X)=\frac{\min(X)+\max(X)}{2}}$

Estimator 2: $\hat{\mu}_2(X)=median(X)$

\vspace{3mm}

Use Monte Carlo methods to estimate the standard error for both estimators, for values of $\mu$ on $[-5,5]$ 
\end{frame}

\end{document} 


