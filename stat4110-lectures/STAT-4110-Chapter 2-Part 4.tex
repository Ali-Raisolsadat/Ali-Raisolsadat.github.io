%========================
% Document class and theme
%========================
\documentclass[8pt]{beamer}
\usetheme[progressbar=frametitle]{metropolis}
\setbeamersize{text margin left=10mm, text margin right=10mm}
\usepackage{appendixnumberbeamer} % appendix slide numbering
\setbeamertemplate{theorems}[numbered]

%========================
% Core packages
%========================
\usepackage{amsmath, amsfonts, amssymb, amsthm} % math + theorems
\usepackage{booktabs}        % professional tables
\usepackage{hyperref}        % hyperlinks
\usepackage{xcolor}          % colors
\usepackage{xspace}          % spacing for custom commands

%========================
% Algorithms
%========================
\usepackage{algorithm}
\usepackage{algpseudocode}
\newtheorem{proposition}{Proposition}
\usepackage{bbm}


%========================
% Plots and TikZ
%========================
\usepackage{pgfplots}
\usepgfplotslibrary{dateplot}
\usepackage{tikz}
\usetikzlibrary{positioning}

% ==========================================
% Professional Code Listing Setup
% ==========================================
\usepackage{listings}
\definecolor{codegreen}{rgb}{0,0.6,0}
\definecolor{codegray}{rgb}{0.5,0.5,0.5}
\definecolor{codepurple}{rgb}{0.58,0,0.82}
\definecolor{backcolour}{rgb}{0.95,0.95,0.92}

\lstdefinestyle{mystyle}{
    backgroundcolor=\color{backcolour},   
    commentstyle=\color{codegreen},
    keywordstyle=\color{magenta},
    numberstyle=\tiny\color{codegray},
    stringstyle=\color{codepurple},
    basicstyle=\ttfamily\footnotesize,
    breakatwhitespace=false,         
    breaklines=true,                 
    captionpos=b,                    
    keepspaces=true,                 
    numbers=left,                    
    numbersep=5pt,                  
    showspaces=false,                
    showstringspaces=false,
    showtabs=false,                  
    tabsize=2
}

\lstset{style=mystyle}


%========================
% Custom commands
%========================
\newcommand{\themename}{\textbf{\textsc{metropolis}}\xspace}

%========================
% Custom footline
%========================
\setbeamertemplate{footline}
{%
  \leavevmode%
  \hbox{%
  \begin{beamercolorbox}[wd=.35\paperwidth,ht=2.5ex,dp=1.5ex,center]{author in head/foot}%
    \usebeamerfont{author in head/foot}\insertshortauthor
  \end{beamercolorbox}%
  \begin{beamercolorbox}[wd=.3\paperwidth,ht=2.5ex,dp=1ex,center]{title in head/foot}%
    \usebeamerfont{title in head/foot}\insertshorttitle
  \end{beamercolorbox}%
  \begin{beamercolorbox}[wd=.3\paperwidth,ht=2.5ex,dp=1ex,right]{date in head/foot}%
    \usebeamerfont{date in head/foot}\insertframenumber{} / \inserttotalframenumber
  \end{beamercolorbox}}%
  \vskip0pt%
}

%========================
% Beamer tweaks
%========================
\setbeamertemplate{navigation symbols}{} % remove default navigation symbols


%%%%%%%%%%%%%%%%%%%%%%%%%%%%%%%%%%%%%%%%%%%%%%%%%%%%%%%%%%%%%%%%%%%%
%%%%%%%%%%%%%%%%%%%%%%%%%%%%%%%%%%%%%%%%%%%%%%%%%%%%%%%%%%%%%%%%%%%%
% AQUI SE DEFINEN LAS IMAGENES PARA UTILIZAR DESPUES
%\pgfdeclareimage[interpolate=true, height=7cm,width=16cm]{halton-points}{halton-points}
%\pgfdeclareimage[interpolate=true, height=3cm, width =4cm]
%{serie-petroleo-reducido}{serie-petroleo-reducido}
%\pgfdeclareimage[interpolate=true, height=3cm, width =4cm]{rectangle-triangle}{rectangle-triangle}
%\pgfdeclareimage[interpolate=true, height=3cm, width =4cm]{any-angle}{any-angle}
%\pgfdeclareimage[interpolate=true, height=3cm, width =4cm]{Pythagoras}{Pythagoras}


\title{Chapter 2 - Simulating Statistical Models}
\subtitle{Poisson Processes.}
\author{Prof. Alex Alvarez, Ali Raisolsadat}
\institute{School of Mathematical and Computational Sciences \\ University of Prince Edward Island}
\date{} % leave empty or add \today
%\title[Stat 4110]{Stat 4110 Statistical Simulation}
%\subtitle{}
%\author[University of Prince Edward Island]{School of Mathematical and Computational Sciences \\ University of Prince Edward Island}

%========================
% Begin document
%========================
\begin{document}

%-------------------
% Title frame
%-------------------
\maketitle

%-------------------
% Slide 1: Poisson Processes
%-------------------
\begin{frame}{Poisson Processes}
\textbf{Poisson processes} are typically used to model the occurrence of events in time. More generally, they can also be used to model the occurrence of events in space.

\vspace{2mm}

\textbf{Example}: The arrival times of people to the Emergency Room in a Hospital during a predetermined interval of time $[t_1,t_2]$ is usually modelled as a Poisson Process.

\vspace{2mm}

\textbf{Example}: At a future moment in time, the location of each and every fish of a given species in a lake can also be modelled as a Poisson process.

\vspace{2mm}

In these two examples notice that the number of arrivals/number of fish in the lake is random. Also the  arrival times/location of the fish is random.
\end{frame}

%-------------------
% Slide 2: Poisson Processes
%-------------------
\begin{frame}{Poisson Processes}
In order to study the Poisson Process we need to start first with the Poisson distribution.

\vspace{2mm}

A random variable $X$ has \textbf{Poisson distribution} with parameter $\lambda$ if it takes non-negative integer values and its probability mass function is given by:
$$P(X=k)=e^{-\lambda}\frac{\lambda^k}{k!} \text{ for } k=0,1,2.\ldots$$

\vspace{2mm}

\textbf{Some properties}:

\vspace{2mm}

If $X \sim \text{Pois}(\lambda)$ and $Y\sim \text{Pois}(\mu)$ are independent  then:

\begin{itemize}
	\item $E(X)=\lambda$
	\item $Var(X)=\lambda$
	\item $X+Y \sim \text{Pois}(\lambda+\mu)$
	\end{itemize}
\end{frame}

%-------------------
% Slide 3: Poisson Processes
%-------------------
\begin{frame}{Poisson Processes}
Poisson processes can be defined on very general spaces. For our purposes, we will consider that they are defined on subsets of $ \mathbb{R}^d$. 

\vspace{3mm}

A \textbf{Poisson Process} on a set $D \subseteq \mathbb{R}^d $ with intensity function $\lambda: \mathbb{R}^d \longrightarrow [0,\infty)$ is a random set of points $\Pi \subseteq D$ such that the following two conditions hold:

\begin{enumerate}[a.]
	\item If $A \subseteq D$, then $|\Pi \cap A| \sim Pois (\Lambda(A))$ where $|\Pi \cap A|$ is the number of points of $\Pi$ in $A$ and
	\item If $A, B \subseteq D$ are disjoint, then $|\Pi \cap A|$ and $|\Pi \cap B|$ are independent.
\end{enumerate}
\end{frame}

%-------------------
% Slide 4: Poisson Processes
%-------------------
\begin{frame}{Remarks}
\begin{itemize}
	\item The number of points of the Poisson Process that are located in set $A$ is random, moreover $E(|\Pi \cap A|)=\Lambda(A)$
	\item On average, regions with large values of the intensity function $\lambda$ will have more concentration of points than regions with small values of $\lambda$
	\item In the particular case where the function $\lambda$ is constant over a region, the Poisson Process points are uniformly distributed over that region.
\end{itemize}
\end{frame}

%-------------------
% Slide 5: Poisson Processes
%-------------------
\begin{frame}
Depending on the specific problem there may be different ways to simulate a Poisson process. One of the most straightforward ways to do this is summarized by the following two-step process:

\vspace{2mm}

\textbf{Step 1}: Generation of the number of points \\
\textbf{Step 2}: Generation of the location of the points

\begin{algorithm}[H]
\caption{Generate a Poisson Process}\label{alg:poisson-process}
\begin{algorithmic}[1]
  \State \textbf{Input:} Intensity function $\lambda(\cdot)$, region $D$
  \State Generate $N \sim \text{Poisson}(\Lambda(D))$
  \For{$i = 1$ to $N$}
    \State Generate $X_i \sim 1_D \, \dfrac{\lambda(\cdot)}{\Lambda(D)}$
  \EndFor
  \State \textbf{Output:} Points $\{X_1, X_2, \ldots, X_N\}$ forming a Poisson process on $D$
\end{algorithmic}
\end{algorithm}

\vspace{2mm}

\textbf{Remark}: The density function $1_D \lambda(\cdot)/\Lambda(D)$ is defined as
\begin{equation*}
1_D \lambda(\cdot)/\Lambda(D)=\left\{ 
\begin{array}{ll}  
\lambda(x)/\Lambda(D) & \text{ if } x \in D\\
0 & \text{ if } x \notin D \\
\end{array}\right.
\end{equation*}
\end{frame}

%-------------------
% Slide 6: Poisson Processes
%-------------------
\begin{frame}{Remarks}
\begin{itemize}
	\item The previous algorithm feasibility is linked to our ability to generate 
samples of points in $\mathbb{R}^d$ that follow the density function $1_D \lambda(\cdot)/\Lambda(D)$
	\item Depending on the intensity function $\lambda$ this might be difficult.
	\item In some cases, the use of rejection methods may be necessary.
\end{itemize}
\end{frame}

%-------------------
% Slide 7: Poisson Processes Example
%-------------------
\begin{frame}{Poisson Process Example}
\textbf{Example}: Generate one sample corresponding to a Poisson process with constant intensity $\lambda=1$ on the interval $D=[0,10]$

$\displaystyle{\Lambda(D)=\int_0^{10} \lambda(x) dx=\int_0^{10} 1 \cdot dx=10}$

\begin{algorithm}[H]
\caption{Generate Poisson Random Points}\label{alg:poisson-points}
\begin{algorithmic}[1]
  \State \textbf{Input:} Interval $[0,10]$, intensity $\lambda = 1$
  \State Generate $N \sim \text{Poisson}(10)$
  \For{$i = 1$ to $N$}
    \State Generate $X_i \sim \text{Uniform}(0,10)$
  \EndFor
  \State \textbf{Output:} Set of Poisson points $\{X_1, X_2, \ldots, X_N\}$
\end{algorithmic}
\end{algorithm}
\end{frame}

%-------------------
% Slide : Poisson Processes Example Code
%-------------------
%\begin{frame}[fragile]{Poisson Process Example -- Code}
%\begin{columns}
%\column{0.48\textwidth}
%\textbf{R Code}
%\begin{lstlisting}[language=R]
%N <- rpois(1, 10)
%X <- runif(N, min=0, max=10)
%Y <- replicate(N, 0)
%plot(X, Y)
%\end{lstlisting}
%
%\column{0.48\textwidth}
%\textbf{Python Code}
%\begin{lstlisting}[language=Python]
%N = np.random.poisson(10)
%X = np.random.uniform(0, 10, N)
%Y = np.zeros(N)
%plt.scatter(X, Y)
%plt.show()
%\end{lstlisting}
%\end{columns}

%\vspace{0.5em}

%\textbf{Remark:} A code like this one may give an error if the generated value 
%of $N$ is 0. In that case, no events occur in the given set $D$.
%\end{frame}

%-------------------
% Slide 8: Poisson Process -- Thinning Method Example
%-------------------
\begin{frame}{Poisson Process -- Thinning Method Example}
\textbf{Example}: Generate one sample corresponding to a Poisson process with  intensity $\displaystyle{\lambda(x)=\frac{x}{50}+\frac{3x^2}{100}}$ on the interval $D=[0,15]$ and 0 otherwise.
$\displaystyle{\Lambda(D)=\int_0^{15} \lambda(x) dx=\int_0^{15} \left(\frac{x}{50}+\frac{3x^2}{100}\right) dx=\left.\left( \frac{x^2+x^3}{100} \right) \right|_0^{15}=36}$
\
\vspace{3mm}

Following the previous algorithm we have to generate random numbers that follow the density $\lambda(x)/\Lambda(D)$. In this specific case probably the easiest way to do that is using a rejection algorithm.

\vspace{3mm}

A roughly equivalent method (called thinning method) is described in Algorithm 2.41 from the textbook. The thinning method turns a Poisson process with intensity $\lambda$ into a Poisson process with intensity 
$\lambda^*\leq\lambda$ by rejecting some of the points.
\end{frame}

%-------------------
% Slide 9: Thinning method
%-------------------
\begin{frame}{Thinning Method}
\textbf{Objective:} Generate a realization of a Poisson process with intensity $\lambda^*$. Let $\lambda^*\leq\lambda$ and $\Lambda(D)=\int_D \lambda(x)dx$.

\vspace{2mm}
\begin{algorithm}[H]
\caption{Generate a Nonhomogeneous Poisson Process via Thinning}
\label{alg:nonhom-poisson}
\begin{algorithmic}[1]
  \State \textbf{Input:} Intensity function $\lambda(x)$, upper bound $\lambda^*(x)$, domain $D$
  \State Generate $N \sim \text{Pois}(\Lambda(D))$
  \State Initialize $\Pi \gets \emptyset$
  \For{$i = 1$ to $N$}
    \State Generate $X_i \sim \dfrac{\lambda(x)}{\Lambda(D)}$
    \State Generate $U \sim U[0,1]$
    \If{$U < \dfrac{\lambda^*(X_i)}{\lambda(X_i)}$}
      \State $\Pi \gets \Pi \cup \{X_i\}$
    \EndIf
  \EndFor
  \State \textbf{Output:} $\Pi$ (set of accepted points)
\end{algorithmic}
\end{algorithm}
\end{frame}

%-------------------
% Slide 10: Poisson Process -- Thinning Method Example
%-------------------
\begin{frame}[fragile]{Poisson Process -- Thinning Method Example}
In the previous example $\displaystyle{\lambda(x)=\frac{x}{50}+\frac{3x^2}{100}}$ on $[0,15]$. This is an increasing function so its maximum is achieved at $x=15$, and we have $\lambda(15)=7.05$.

\vspace{2mm}

This means that we could start with a Poisson process with intensity 
$\tilde{\lambda}=7.05 \geq \lambda(x)$ and apply the thinning method.

\vspace{2mm}

$\displaystyle{\tilde{\Lambda}(D)=\int_0^{15} \tilde{\lambda}(x) dx=7.05 \cdot 15=105.75}$
\end{frame}

%-------------------
% Slide : Poisson Process -- Thinning Method Example
%-------------------
%\begin{frame}[fragile]{Poisson Process -- Thinning Method Example}
%\begin{columns}
%\begin{column}{0.48\textwidth}
%\textbf{R Code}
%\begin{lstlisting}[language=R]
%lam_tilde <- 7.05
%a <- 0; b <- 15
%Lambda_tilde <- lam_tilde * (b - a)
%
%N <- rpois(1, Lambda_tilde)
%X <- runif(N, min=0, max=b)
%Z <- replicate(N, 0)
%plot(X, Z)

%R <- X/50 + 3*X^2/100
%U <- runif(N)
%Y <- vector()
%k <- 1
%for (i in c(1:N)) {
 % if (R[i] > lam_tilde * U[i]) {
  %  Y[k] = X[i]
   % k <- k + 1
 % }
%}
%Z1 <- replicate(k-1, 0)
%plot(Y, Z1)
%\end{lstlisting}
%\end{column}

%\begin{column}{0.48\textwidth}
%\textbf{Python Code}
%\begin{lstlisting}[language=Python]
%lam_tilde = 7.05
%a, b = 0, 15
%Lambda_tilde = lam_tilde * (b - a)

%N = np.random.poisson(Lambda_tilde)
%X = np.random.uniform(a, b, N)
%Z = np.zeros(N)
%plt.scatter(X, Z)
%plt.show()

%R = X/50 + 3*X**2/100
%U = np.random.uniform(0, 1, N)
%Y = X[R > lam_tilde * U]
%Z1 = np.zeros(len(Y))

%plt.scatter(Y, Z1, color="red")
%$plt.show()
%\end{lstlisting}
%\end{column}
%\end{columns}
%\end{frame}

%-------------------
% Slide 11: Bivariate Poisson Process Example
%-------------------
\begin{frame}{Bivariate Poisson Process Example}
\textbf{Example}: Generate one sample corresponding to a Poisson process with  intensity 
$\lambda(x_1,x_2)=500x_1$ on the rectangle $D=[0,2]\times[0,1]$

$$\displaystyle{\Lambda(D)=\int_0^1\int_0^2 500 x_1 dx_1 dx_2=1000}$$

This means that we will have to generate points on $[0,2]\times[0,1]$ according to the density $\displaystyle{f(x_1,x_2)=\frac{\lambda(x_1,x_2)}{\Lambda(D)}=\frac{x_1}{2}}$ on $D$ and 0 outside of $D$.

From here we can see that $f(x_1,x_2)$ can be written as the product of  $f_1(x_1)=x_1/2$ and $f_2(x_2)=1$. Notice that $f_1$ is a density function on $[0,2]$ and $f_2$ is a density function on $[0,1]$.

\vspace{2mm}

The generation of random vectors with density $f$ can be done by {\bf independently} generating its components according to densities $f_1$ and $f_2$ respectively
\end{frame}

%-------------------
% Slide 12: Bivariate Poisson Process Algorithm
%-------------------
\begin{frame}{Algorithm: Generating Bivariate Poisson Process}
\begin{algorithm}[H]
\caption{Generate Bivariate Poisson Process}
\begin{algorithmic}[1]
  \State \textbf{Input:} $\Lambda(D) = 1000$
  \State Generate $N \sim \text{Pois}(1000)$
  \For{$i = 1$ to $N$}
    \State Generate $X_1[i] \sim f_1$
    \State Generate $X_2[i] \sim f_2$
  \EndFor
  \State \textbf{Output:} $X = (X_1, X_2)$
\end{algorithmic}
\end{algorithm}

\vspace{2mm}

\textbf{Remarks}
\begin{itemize}
	\item To generate random numbers according to density $f_1$ we can use the inverse transform method
	\item $f_2$ is the uniform distribution density on $[0,1]$
\end{itemize}
\end{frame}

%-------------------
% Slide : Bivariate Poisson Process Example -- Code
%-------------------
%\begin{frame}[fragile]{Bivariate Poisson Process Example}
%\begin{columns}[t]
%\column{0.48\textwidth}
%\textbf{R Code}
%\begin{lstlisting}[language=R]
%N <- rpois(1, 200)
%U1 <- runif(N, min=0, max=1)
%X1 <- sqrt(4 * U1)
%X2 <- runif(N, min=0, max=1)
%plot(X1, X2)
%\end{lstlisting}

%\column{0.48\textwidth}
%\textbf{Python Code}
%\begin{lstlisting}[language=Python]
%N = np.random.poisson(200)
%U1 = np.random.uniform(0, 1, N)
%X1 = np.sqrt(4 * U1)
%X2 = np.random.uniform(0, 1, N)

%plt.scatter(X1, X2)
%plt.show()
%\end{lstlisting}
%\end{columns}
%\end{frame}

%-------------------
% Slide : Bivariate Poisson Process Example
%-------------------
%\begin{frame}[fragile]{Bivariate Poisson Process Example}
%\begin{center}
%\includegraphics[scale=0.6]{Poisson-process-plane.png}
%\end{center}
%\end{frame}

%-------------------
% Slide 13: Homework
%-------------------
\begin{frame}{Homework}
Generate one sample corresponding to a Poisson process with  intensity 
$\lambda(x_1,x_2)=30(x_1^2+x_2^2)$ on the rectangle $D=[0,3]\times[0,4]$

\vspace{3mm}

\textbf{Hint}: Use the thinning method
\end{frame}

\end{document}