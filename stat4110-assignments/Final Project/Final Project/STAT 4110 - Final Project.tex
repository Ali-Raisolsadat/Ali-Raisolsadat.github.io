\documentclass[11pt]{article}

\usepackage{amsmath, amssymb, amsfonts}
\usepackage{geometry}
\geometry{margin=1in}
\usepackage{setspace}
\usepackage{hyperref}
\usepackage{enumitem}
\usepackage{graphicx}
\usepackage{booktabs}
\usepackage{mathtools}

\title{\textbf{Final Project \\ STAT 4110 Statistical Simulation \\ Winter 2026 Semester}}
\author{School of Mathematical and Computational Sciences \\ University of Prince Edward Island \\ Ali Raisolsadat}

\begin{document}
\maketitle

\section*{Overview}

This project is designed to familiarize students with real-world applications of statistical simulation, stochastic modelling, and analytical reasoning. The available options for the final project may involve developing a computational model, implementing simulation techniques covered in class, and using simulation-based or analytical methods to study system behaviour. Students are free to use any dataset or programming language of their choice, unless otherwise specified.

Choose \textbf{one} of the following three project directions:

\begin{enumerate}
	\item Use your modelling skills to model and implement \textbf{one of the following four} real world problems:
	\begin{itemize}
		\item Emergency Room Patient Arrivals -- Poisson and Markov Chain Simulation
    		\item Numerical Option Pricing: Crank--Nicolson PDE vs. Monte Carlo Simulation
    		\item Bayesian Neural Networks using Markov Chain Monte Carlo
    		\item Your very own choice of modelling problem 
	\end{itemize}
    \item Replicating the full design and implementation of \textbf{one} published paper related to our course material
    \item Literature review of \textbf{5--15 papers} on a topic related to the course
\end{enumerate}

Projects will be completed in teams of three members. Each team will be responsible for defining the scope, data sources, and motivation of their work.

\subsection*{Deliverables}

For any of the project choices, the deliverables are as follows:

\begin{enumerate}
    \item \textbf{Introductory Summary} (2 pages max, 10/30 points):  
    Includes a brief description of the chosen project, references and papers that the group will use, and the roles of each student in the group.    
    \item \textbf{Summary Report} (5 pages max, 20/30 points): A detailed report summarizing the approach, implementation, results, and discussion. Include any supporting figures, tables, or appendices as needed. You are also responsible to attach a ZIP file containing all code used in the project. If you are using Git (GitHub or GitLab), please add me as the owner, so I can review your code. 
\end{enumerate}

\textbf{Disclaimer}: The use of LLMs, including GPT, is permitted ONLY for coding assistance. Any use of such tools must be explicitly acknowledged in the report.


\newpage

% --------------------------------------------------------
\section*{Project Choice 1: Modelling Option}

\subsection*{Purpose}
The purpose of this project choice is to give you hands-on experience in developing and analysing statistical/mathematical models for real-world problems. By working on a modelling-based project you will learn how to formalize assumptions, implement computational methods and interpret model outcomes. This choice emphasizes on using the material learnt throughout the course and understanding the behaviour of complex systems under uncertainty.

\subsection*{Learning Objectives}
\begin{itemize}
    \item Design and implement a computational model for a selected application.
    \item Apply simulation or numerical methods to approximate solutions when analytic solutions are impossible.
    \item Compare model predictions with alternative approaches or benchmarks.
    \item Communicate methodology, results, and insights effectively through written reports and presentations.
\end{itemize}

\subsection*{Scope}
You are expected to choose \textbf{one of the four} modelling-based problems:
	\begin{enumerate}
		\item Emergency Room Patient Arrivals -- Poisson and Markov Chain Simulation
    		\item Numerical Option Pricing: Crank--Nicolson PDE vs. Monte Carlo Simulation
    		\item Bayesian Neural Networks using Markov Chain Monte Carlo
    		\item Your very own choice of modelling problem 
	\end{enumerate}
As a team, you are responsible for selecting appropriate datasets, defining the project scope, and documenting their methodology and results.

\newpage

% --------------------------------------------------------
\subsection*{Choice 1: Emergency Room Patient Arrivals -- Poisson and Markov Chain Simulation}

\subsubsection*{1. Motivation}

Hospitals need to manage patient flow in the Emergency Room (ER) to ensure quality care and optimal resource allocation. Patient arrivals are random and the ER load changes over time. Modelling this system helps answer:

\begin{itemize}
    \item How does ER load evolve over time based on patient arrivals?
    \item How many patients arrive under different ER congestion states?
    \item What periods of high congestion are likely, and how can hospitals prepare?
\end{itemize}

\subsubsection*{2. Basic Concepts}

We model the ER using:

\begin{itemize}
    \item A \textbf{Markov chain} representing ER load states:
    \begin{itemize}
        \item Low load (L)
        \item Moderate load (M)
        \item High load (H)
    \end{itemize}
    \item A \textbf{Poisson process} for patient arrivals:
    \begin{itemize}
        \item Arrival rate depends on the current ER load: $\lambda_L, \lambda_M, \lambda_H$
        \item Number of arrivals in a time interval is random
    \end{itemize}
\end{itemize}

\subsubsection*{3. Markov Chain Model of ER Load}

\begin{itemize}
    \item Define a transition matrix for the ER states. Example:
    \[
        P = 
        \begin{bmatrix}
        P_{LL} & P_{LM} & P_{LH} \\
        P_{ML} & P_{MM} & P_{MH} \\
        P_{HL} & P_{HM} & P_{HH} 
        \end{bmatrix}
    \]
    \item Each row represents probabilities of moving to another state in the next hour.
    \item The Markov chain determines how the ER load changes over time.
\end{itemize}

\subsubsection*{4. Poisson Process for Patient Arrivals}

\begin{itemize}
    \item Patient arrivals are modelled as a Poisson process with rate $\lambda$ dependent on current ER load.
    \item For a fixed interval (e.g., 1 hour), number of arrivals $N \sim \text{Poisson}(\lambda)$.
    \item Arrival times within the interval can be simulated using uniform random numbers.
\end{itemize}

\subsubsection*{5. Linking Markov Chain and Poisson Process}

\begin{itemize}
    \item Simulate the ER state over time using the Markov chain.
    \item For each time interval, generate arrivals according to the Poisson rate corresponding to the current ER state.
    \item Record both ER state and arrival times for analysis.
\end{itemize}

\subsubsection*{6. Analysis and Visualization}

\begin{itemize}
    \item Plot ER load state over time.
    \item Plot number of patient arrivals per interval.
    \item Compute expected vs observed arrivals per ER state.
    \item Identify peak congestion periods.
\end{itemize}

\subsubsection*{7. Suggested Experiments}

\begin{itemize}
    \item Compare results for different transition matrices (e.g., congestion more likely to persist vs quickly recover).
    \item Test sensitivity to Poisson rates ($\lambda_L, \lambda_M, \lambda_H$).
    \item Optional: introduce time-of-day effects (inhomogeneous Poisson process).
    \item Optional: extend to continuous-state Markov chain representing exact patient counts.
\end{itemize}

\subsubsection*{8. Example Parameters}

\begin{center}
\begin{tabular}{|c|c|}
\hline
Parameter & Suggested Value \\ \hline
ER load states & Low, Moderate, High \\
Transition probabilities & Adjust realistically, e.g., $P_{HH}=0.6$, $P_{LL}=0.7$ \\
Poisson rates $\lambda$ & Low=2, Moderate=5, High=10 patients/hour \\
Simulation period & 1 week (hourly intervals) \\ \hline
\end{tabular}
\end{center}

\newpage

% --------------------------------------------------------
\subsection*{Option 2: Black--Scholes Model, Monte Carlo Pricing, and Crank--Nicolson Finite Differences}

\subsubsection*{1. What is the Black--Scholes model and why use it?}

The Black--Scholes model is a continuous-time stochastic model for the evolution of a financial asset price \(S_t\). It assumes that the asset price follows a geometric Brownian motion (GBM):
\[
dS_t = \mu S_t\,dt + \sigma S_t\,dW_t
\]
where
\begin{itemize}
  \item \(\mu\) is the instantaneous expected return (drift)
  \item \(\sigma>0\) is the volatility (instantaneous standard deviation)
  \item \(W_t\) is a standard Brownian motion
\end{itemize}

The model is used because (i) it is analytically tractable in many cases, (ii) under risk-neutral valuation it leads to a linear parabolic PDE for derivative prices, and (iii) it captures multiplicative stochastic growth and log-normal marginal distributions for \(S_t\). The Black--Scholes framework underpins much of modern quantitative finance and provides a baseline for comparing numerical methods (PDE solvers, Monte Carlo, lattice models, etc.).

\subsubsection*{2. Risk-neutral pricing and derivation of the Black--Scholes PDE}

Consider a derivative security with payoff at maturity \(T\) given by \(\Phi(S_T)\). The \emph{no-arbitrage} (risk-neutral) pricing formula states that the arbitrage-free price at time \(t\) is the discounted risk-neutral expectation:
\[
V(S_t,t) = e^{-r (T-t)} \mathbb{E}^{\mathbb{Q}}\big[\,\Phi(S_T)\mid S_t\,\big]
\]
where \(r\) is the risk-free rate and \(\mathbb{Q}\) denotes the risk-neutral measure. Under the risk-neutral measure the asset dynamics become
\[
dS_t = r S_t\, dt + \sigma S_t\, dW_t^{\mathbb{Q}}
\]

Applying Itô's formula to \(V(S_t,t)\) gives
\[
dV = \left( \frac{\partial V}{\partial t} + rS \frac{\partial V}{\partial S} + \tfrac{1}{2}\sigma^2 S^2 \frac{\partial^2 V}{\partial S^2} \right) dt
      + \sigma S \frac{\partial V}{\partial S} dW_t^{\mathbb{Q}}
\]
Under risk-neutral valuation, the discounted derivative price must be a martingale, which leads to the Black--Scholes backward PDE:
\[
\boxed{\ \frac{\partial V}{\partial t} + \tfrac{1}{2}\sigma^2 S^2 \frac{\partial^2 V}{\partial S^2} + r S \frac{\partial V}{\partial S} - r V = 0 \ }
\]
Terminal condition: \(V(S,T)=\Phi(S)\).

\subsubsection*{3. Probabilistic (Monte Carlo) representation}

The derivative price can be approximated by the risk-neutral expectation:
\[
V(S_0,0) = e^{-rT} \mathbb{E}^{\mathbb{Q}}[\Phi(S_T)\mid S_0 ]
\]
Under GBM,
\[
S_T = S_0 \exp\Big( \big(r - \tfrac{1}{2}\sigma^2\big)T + \sigma \sqrt{T}\,Z \Big),\qquad Z\sim \mathcal{N}(0,1)
\]

\subsubsection*{3.1 Euler--Maruyama discretization}

Discretize the interval \([0,T]\) into \(N\) steps \(\Delta t = T/N\). The Euler--Maruyama scheme:
\[
S_{n+1} = S_n + r S_n \Delta t + \sigma S_n \sqrt{\Delta t} Z_n \quad Z_n \sim N(0,1)
\]
or the log-Euler form:
\[
S_{n+1} = S_n \exp\Big( (r - \tfrac{1}{2}\sigma^2)\Delta t + \sigma \sqrt{\Delta t}\, Z_n \Big)
\]

\subsubsection*{3.2 Monte Carlo pricing of European options}

Simulate \(M\) paths \(S_T^{(m)}\) and compute:
\[
V_0 \approx e^{-rT} \frac{1}{M} \sum_{m=1}^{M} \Phi\big(S_T^{(m)}\big)
\]

\paragraph{Remarks:}  
\begin{itemize}
    \item Euler--Maruyama is first-order accurate in \(\Delta t\) for strong convergence.  
    \item Monte Carlo converges as \(O(M^{-1/2})\); variance reduction can improve efficiency.  
    \item Flexible for path-dependent options and multi-asset problems.
\end{itemize}

\subsubsection*{4. Numerical solution of the PDE: Crank--Nicolson finite differences}

Finite-difference methods discretize the PDE on a grid in asset price \(S\) and time \(t\). Crank--Nicolson (CN) is a widely used implicit scheme that is second-order accurate in both space and time and is unconditionally stable (for the linear parabolic PDEs we consider).

Below we derive the CN discretization and obtain the tridiagonal linear system that must be solved at each time step.

\subsubsection*{4.1 Domain and grid}

Truncate the infinite spatial domain \(S\in(0,\infty)\) to a large finite domain \(S\in[S_{\min},S_{\max}]\). A common choice is \(S_{\min}=0\) and \(S_{\max}=L S_0\) with \(L=3\) or \(4\), chosen so the option value is insensitive to further increases in \(S_{\max}\).

Define a uniform spatial grid with \(M+1\) nodes:
\[
S_i = S_{\min} + i\Delta S \qquad \Delta S = \frac{S_{\max}-S_{\min}}{M} \quad i=0,1,\dots,M
\]
Define a temporal grid stepping backward from \(T\) to \(0\) with \(N\) time-steps:
\[
t^n = n\Delta t \qquad \Delta t = \frac{T}{N} \quad n=0,1,\dots,N
\]
We will denote the numerical approximation to \(V(S_i, t^n)\) by \(V_i^n\). (Because the PDE is typically integrated backward in time from \(t=T\) to \(t=0\), many authors index time reversely; here we take \(n=0\) for time \(t=0\) and note the CN update moves from known \(V^n\) to \(V^{n+1}\).)

\subsubsection*{4.2 Spatial derivatives approximation}

Approximate first and second derivatives by centered finite differences:
\[
\frac{\partial V}{\partial S}\bigg|_{S_i t^n} \approx \frac{V_{i+1}^n - V_{i-1}^n}{2\Delta S}
\qquad
\frac{\partial^2 V}{\partial S^2}\bigg|_{S_i t^n} \approx \frac{V_{i+1}^n - 2V_i^n + V_{i-1}^n}{(\Delta S)^2}
\]

\subsubsection*{4.3 Semi-discrete operator}

Plugging these into the PDE, at grid node \((S_i,t^n)\) the differential operator
\[
\mathcal{L}V := \tfrac{1}{2}\sigma^2 S^2 V_{SS} + r S V_S - r V
\]
is approximated by
\[
(\mathcal{L}V)_i^n \approx \tfrac{1}{2}\sigma^2 S_i^2 \frac{V_{i+1}^n - 2V_i^n + V_{i-1}^n}{(\Delta S)^2}
+ r S_i \frac{V_{i+1}^n - V_{i-1}^n}{2\Delta S} - r V_i^n
\]

It is convenient to rewrite this as a linear combination
\[
(\mathcal{L}V)_i^n \approx \alpha_i^n V_{i-1}^n + \beta_i^n V_i^n + \gamma_i^n V_{i+1}^n
\]
with coefficients (dropping the explicit time index for the coefficients since they depend only on \(S_i\)):
\[
\begin{aligned}
\alpha_i &:= \frac{1}{2}\sigma^2 S_i^2 \frac{1}{(\Delta S)^2} - \frac{r S_i}{2\Delta S} \\[4pt]
\beta_i &:= -\frac{1}{2}\sigma^2 S_i^2 \frac{2}{(\Delta S)^2} - r \\[4pt]
\gamma_i &:= \frac{1}{2}\sigma^2 S_i^2 \frac{1}{(\Delta S)^2} + \frac{r S_i}{2\Delta S}
\end{aligned}
\]
Thus
\[
(\mathcal{L}V)_i^n \approx \alpha_i V_{i-1}^n + \beta_i V_i^n + \gamma_i V_{i+1}^n
\]

\subsubsection*{4.4 Crank--Nicolson time discretization}

Crank--Nicolson averages the spatial operator at times \(t^n\) and \(t^{n+1}\). The time derivative is approximated by
\[
\frac{V_i^{n+1} - V_i^{n}}{\Delta t} \approx \frac{1}{2}\big( (\mathcal{L}V)_i^{n+1} + (\mathcal{L}V)_i^{n} \big).
\]
Rearrange to obtain the CN update equation:
\[
V_i^{n+1} - \frac{\Delta t}{2}(\mathcal{L}V)_i^{n+1}
= V_i^{n} + \frac{\Delta t}{2}(\mathcal{L}V)_i^{n}
\]

Substituting the finite-difference representation of \(\mathcal{L}\) yields a linear relation for interior nodes \(i=1,\dots,M-1\):
\[
V_i^{n+1} - \frac{\Delta t}{2}\big( \alpha_i V_{i-1}^{n+1} + \beta_i V_i^{n+1} + \gamma_i V_{i+1}^{n+1} \big)
= V_i^{n} + \frac{\Delta t}{2}\big( \alpha_i V_{i-1}^{n} + \beta_i V_i^{n} + \gamma_i V_{i+1}^{n} \big)
\]

Collect terms for \(V_{i-1}^{n+1}\), \(V_i^{n+1}\), \(V_{i+1}^{n+1}\) on the left-hand side and known \(V^{n}\) terms on the right-hand side. Define coefficients:
\[
\begin{aligned}
A_i &:= -\frac{\Delta t}{2}\alpha_i \\[4pt]
B_i &:= 1 - \frac{\Delta t}{2}\beta_i\\[4pt]
C_i &:= -\frac{\Delta t}{2}\gamma_i
\end{aligned}
\qquad\text{(left-hand side coefficients)}
\]
and
\[
\begin{aligned}
\widetilde{A}_i &:= \frac{\Delta t}{2}\alpha_i\\[4pt]
\widetilde{B}_i &:= 1 + \frac{\Delta t}{2}\beta_i\\[4pt]
\widetilde{C}_i &:= \frac{\Delta t}{2}\gamma_i
\end{aligned}
\qquad\text{(right-hand side coefficients)}
\]

Then the CN equation becomes, for each interior \(i\),
\[
A_i V_{i-1}^{n+1} + B_i V_i^{n+1} + C_i V_{i+1}^{n+1}
= \widetilde{A}_i V_{i-1}^{n} + \widetilde{B}_i V_i^{n} + \widetilde{C}_i V_{i+1}^{n}
\]

\subsubsection*{4.5 Matrix form}

Stack the interior unknowns into a vector
\[
\mathbf{V}^{n} = \begin{bmatrix} V_1^{n} \\ V_2^{n} \\ \vdots \\ V_{M-1}^{n} \end{bmatrix}
\]
For each time-step the system can be written compactly as
\[
\boxed{ \quad A\,\mathbf{V}^{\,n+1} \;=\; B\,\mathbf{V}^{\,n} \;+\; \mathbf{d}^{\,n} \quad }
\]
where \(A\) and \(B\) are \((M-1)\times(M-1)\) tridiagonal matrices with entries given by the \(A_i,B_i,C_i\) and \(\widetilde{A}_i,\widetilde{B}_i,\widetilde{C}_i\) coefficients, and \(\mathbf{d}^n\) is a vector that collects boundary contributions. Explicitly, \(A\) has the form
\[
A = \begin{bmatrix}
B_1 & C_1 & 0   & \cdots & 0 \\
A_2 & B_2 & C_2 & \ddots & \vdots \\
0   & \ddots & \ddots & \ddots & 0 \\
\vdots & \ddots & A_{M-2} & B_{M-2} & C_{M-2}\\
0 & \cdots & 0 & A_{M-1} & B_{M-1}
\end{bmatrix}
\]
and \(B\) is similarly
\[
B = \begin{bmatrix}
\widetilde{B}_1 & \widetilde{C}_1 & 0 & \cdots & 0 \\
\widetilde{A}_2 & \widetilde{B}_2 & \widetilde{C}_2 & \ddots & \vdots \\
0 & \ddots & \ddots & \ddots & 0 \\
\vdots & \ddots & \widetilde{A}_{M-2} & \widetilde{B}_{M-2} & \widetilde{C}_{M-2}\\
0 & \cdots & 0 & \widetilde{A}_{M-1} & \widetilde{B}_{M-1}
\end{bmatrix}
\]
The boundary vector \(\mathbf{d}^n\) accounts for known values \(V_0^n\) and \(V_M^n\) appearing in the finite-difference stencils for the first and last interior nodes. For example, the first equation (for \(i=1\)) includes terms proportional to \(V_0^{n}\) and \(V_0^{n+1}\); move these to the right-hand side to obtain the appropriate entry in \(\mathbf{d}^n\).

\paragraph{Boundary conditions.} Reasonable boundary conditions for European call options:
\[
\begin{aligned}
V(S_{\min}=0,t) &= 0 \qquad\text{(call is worthless at zero)}\\
V(S_{\max},t) &\approx S_{\max} - K e^{-r(T-t)}
\end{aligned}
\]
Use these values to compute contributions to \(\mathbf{d}^{n}\). If one uses \(S_{\min}=0\) then \(V_0^n=0\) simplifies the first equation. For the last equation, known \(V_M^n\) provides a known term on the right-hand side.

\subsubsection*{4.6 Solving the tridiagonal system (Thomas algorithm)}

At each time-step we must solve the tridiagonal linear system
\[
A\,\mathbf{V}^{\,n+1} = \mathbf{r}^{\,n}
\qquad \text{where } \mathbf{r}^{\,n} = B\,\mathbf{V}^{\,n} + \mathbf{d}^{\,n}
\]
Because \(A\) is tridiagonal and diagonally dominant for reasonable grids and parameters, the Thomas algorithm (tridiagonal matrix algorithm) solves this in \(O(M)\) operations with two sweeps:

\begin{enumerate}
  \item \emph{Forward elimination:} modify diagonal and right-hand side entries to eliminate subdiagonal entries.
  \item \emph{Backward substitution:} compute unknowns \(V_{M-1}^{n+1},V_{M-2}^{n+1},\dots,V_1^{n+1}\).
\end{enumerate}

This yields an efficient solver and is typically preferred to general-purpose LU decompositions for 1-D PDEs.

\subsubsection*{5. Implementation notes and practical considerations}

\paragraph{Defining the option and model.} Choose a standard option (e.g., European call or put) and clearly define all parameters ($S_0$, $K$, $r$, $\sigma$, $T$, dividends). For European options, the analytical Black--Scholes solution can serve as a reference.

\paragraph{Accuracy comparison.}
\begin{itemize}
\item \textbf{Monte Carlo (MC):} Simulate $N$ paths of the underlying asset, compute discounted payoff averages, and measure absolute or relative error against the analytical solution. Optionally, report standard error of the estimator.
\item \textbf{Crank--Nicolson (CN):} Construct a time--price grid ($\Delta t$, $\Delta S$) and solve the PDE. Compute the error relative to the analytical or highly accurate numerical solution.
\end{itemize}
Compare errors for both methods and observe convergence under grid refinement (CN) or increasing number of paths (MC).

\paragraph{Efficiency comparison.}
\begin{itemize}
\item \textbf{Monte Carlo:} Runtime scales linearly with the number of simulated paths. Error decreases as $1/\sqrt{N}$. Record CPU time for each $N$.
\item \textbf{Crank--Nicolson:} Runtime scales with $M \times N$, where $M$ is the number of time steps and $N$ the number of spatial grid points. Finer grids improve accuracy faster than MC for the same runtime in low-dimensional problems.
\end{itemize}
Generate work--precision plots (error vs CPU time) to visualize efficiency trade-offs.

\paragraph{Practical considerations.}
\begin{itemize}
\item MC is more suitable for high-dimensional or path-dependent options; variance reduction techniques improve efficiency.
\item CN is efficient for low-dimensional options and can handle early exercise (with modifications). Grid resolution should balance accuracy and computational cost.
\item Ensure discretization consistency for non-uniform grids or coordinate transforms.
\end{itemize}


 
\subsubsection*{6. Tasks}

You are asked to implement both Monte Carlo and Crank--Nicolson methods for European call options and perform a comparative study.

\paragraph{1. Constants table (suggested values):}

\begin{center}
\begin{tabular}{|c|c|}
\hline
Parameter & Value \\ \hline
Initial asset price \(S_0\) & 100 \\ 
Strike \(K\) & 100 \\ 
Time to maturity \(T\) & 1 year \\ 
Volatility \(\sigma\) & 0.2, 0.4 \\ 
Risk-free rate \(r\) & 0.01, 0.05 \\ 
Number of time steps \(N\) & 100 \\ 
Number of Monte Carlo paths \(M\) & 50,000 \\ \hline
\end{tabular}
\end{center}

\paragraph{2. Experiments:}
\begin{itemize}
  \item Compute option prices for all combinations of \(\sigma\) and \(r\).  
  \item Compare Monte Carlo estimated prices with Crank--Nicolson numerical PDE prices.  
  \item Measure CPU time and compare efficiency of Monte Carlo vs CN.  
  \item Plot option price vs volatility, option price vs interest rate.  
  \item Optionally, explore variance reduction techniques in Monte Carlo.  
\end{itemize}

\paragraph{3. Deliverables:}
\begin{itemize}
  \item Implementations of Monte Carlo and Crank--Nicolson solvers.  
  \item Table of results for varying \(\sigma\) and \(r\).  
  \item Analysis comparing accuracy and efficiency of the two methods.  
\end{itemize}


\newpage


% --------------------------------------------------------
\subsection*{Option 3: Bayesian Neural Networks via MCMC}
\subsubsection*{1. Background and Motivation}

Traditional neural networks optimize weights to single point estimates, usually via gradient descent. This approach does not quantify uncertainty, which is important when:

\begin{itemize}
    \item making decisions in risk-sensitive applications (medicine, finance, engineering)
    \item detecting anomalies or rare events
    \item forecasting with small datasets
    \item requiring confidence intervals or credible predictions
\end{itemize}

Bayesian neural networks (BNNs) treat the network weights as random variables with prior distributions. The posterior distribution of the weights given observed data allows us to capture uncertainty:

\[
p(w \mid X, y) = \frac{p(y \mid X, w)\, p(w)}{p(X,y)},
\]

where \(w\) are the weights, \(X\) the inputs, \(y\) the observed outputs, \(p(w)\) the prior, and \(p(y \mid X,w)\) the likelihood.

Because the posterior is usually intractable, we approximate it using Markov Chain Monte Carlo (MCMC) methods such as:

\begin{itemize}
    \item Metropolis--Hastings
    \item Gibbs sampling
    \item Hamiltonian Monte Carlo (HMC)
\end{itemize}

\subsubsection*{2. Model Structure}

Consider a simple feedforward neural network with input \(x\), one hidden layer of \(H\) neurons, and output \(y\). Denote the network function as \(f(x; w)\).

\paragraph{Priors:}  
Assign independent Gaussian priors to weights and biases:

\[
w \sim \mathcal{N}(0, \sigma_w^2), \quad b \sim \mathcal{N}(0, \sigma_b^2).
\]

\paragraph{Likelihood (for regression):}  
Assume Gaussian noise on outputs:

\[
y \mid x, w \sim \mathcal{N}(f(x; w), \sigma^2).
\]

\paragraph{Posterior:}  
The posterior distribution of the weights is proportional to:

\[
p(w \mid X, y) \propto p(y \mid X, w) p(w).
\]

\paragraph{Posterior predictive distribution:}  
For a new input \(x_\text{new}\), approximate the predictive distribution by averaging over posterior samples:

\[
\hat{y}(x_\text{new}) \approx \frac{1}{S} \sum_{s=1}^{S} f(x_\text{new}; w^{(s)}),
\]

where \(w^{(1)}, \dots, w^{(S)}\) are sampled from the posterior using MCMC.

\subsubsection*{3. Datasets}

You can use any dataset, but it is recommended to choose \textbf{small or simple datasets} so that MCMC sampling completes in reasonable time. Examples include:

\begin{itemize}
    \item \textbf{Tabular / regression:} Boston Housing, Diabetes, or synthetic datasets such as $y = \sin(x) + \epsilon$.
    \item \textbf{Time series:} Stock prices (subset), daily sales, or temperature data.
    \item \textbf{Classification / small images:} Iris dataset, or a subset of MNIST (e.g., 1000--5000 images).
\end{itemize}

The key is to keep the dataset size manageable for a few thousand MCMC iterations.

\subsubsection*{4. Suggested Experiments}

\begin{itemize}
    \item Vary the prior variance \(\sigma_w^2\) and observe the effect on predictive uncertainty.
    \item Change the noise variance \(\sigma^2\) and compare predictive intervals.
    \item Compare performance and uncertainty estimates between MCMC BNN and standard NN.
    \item Optional: Vary the number of hidden neurons or layers and observe convergence of MCMC samples.
\end{itemize}

\subsubsection*{5. Student Tasks}

\begin{enumerate}
    \item Implement a Bayesian neural network using MCMC.
    \item Generate posterior samples for the weights and biases.
    \item Compute posterior predictive distributions for test inputs.
    \item Investigate how predictions and uncertainty change when:
    \begin{itemize}
        \item the prior variance changes
        \item the output noise changes
        \item the network size changes
    \end{itemize}
    \item Compare accuracy and efficiency with a standard neural network.
\end{enumerate}

\newpage

% --------------------------------------------------------
\section*{Option 2: Replicating the Design and Implementation of a Published Paper}

\subsection*{Background and Motivation}

Replication is a core aspect of scientific research. By attempting to reproduce the results of a published study, you gain a deeper understanding of modelling assumptions, computational methods, and analysis techniques. This option allows you to connect course concepts to real-world research and critically evaluate methodological choices.

\subsection*{Project Task}

You will:

\begin{itemize}
    \item Select a published paper related to stochastic modelling, Monte Carlo and/or Monte Carlo simulation or other topics covered in the course.
    \item Carefully review the paper, focusing on the problem statement, assumptions, model structure, and computational approach.
    \item Implement the methods or experiments described in the paper using a programming language of their choice.
    \item Compare your results with the published results and analyse any discrepancies.
    \item Document challenges encountered, adaptations made, and lessons learned from the replication process.
\end{itemize}

\subsection*{Learning Objectives}

\begin{itemize}
    \item Understand the translation of theoretical models into practical computational methods.
    \item Critically evaluate published results and methodologies.
    \item Develop practical skills in programming, numerical methods, and model validation.
    \item Gain experience in scientific documentation and reproducibility.
\end{itemize}

\newpage

% --------------------------------------------------------
\section*{Option 3: Literature Review of 5--15 Papers}

\subsection*{Background and Motivation}

A literature review allows you to synthesize knowledge on a particular topic, identify trends, gaps, and open questions, and develop critical evaluation skills. This project option emphasizes scholarly analysis rather than implementation, and is ideal for you interested in exploring broader research directions.

\subsection*{Project Task}

You will:

\begin{itemize}
    \item Select a coherent topic or research question related to the course (e.g., stochastic simulations, Bayesian inference, neural network modelling, queueing systems).
    \item Search for and read 5--15 relevant academic papers, preprints, or technical reports.
    \item Summarize key methods, results, assumptions, and conclusions of each paper.
    \item Compare approaches, highlight similarities and differences, and identify gaps in the literature.
    \item Discuss potential future directions or applications suggested by the literature.
\end{itemize}

\subsection*{Learning Objectives}

\begin{itemize}
    \item Develop skills in critical reading, synthesis, and scholarly writing.
    \item Identify research trends and gaps within a focused area.
    \item Learn to communicate complex ideas clearly in written and oral form.
    \item Gain familiarity with academic databases, citation practices, and literature evaluation techniques.
\end{itemize}

\end{document}
